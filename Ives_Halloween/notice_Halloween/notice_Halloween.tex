%\begin{flushleft}
\documentclass[11pt,a4paper,bibliography=totoc,numbers=noenddot,twoside]{scrreprt}%{scrbook} {scrreprt}
%\usepackage[top=2.5 cm, bottom=3 cm, inner=2.5 cm, outer=2 cm]{geometry}

% XeLaTeX can use any Mac OS X font. See the setromanfont command below.
% Input to XeLaTeX is full Unicode, so Un%icode characters can be typed directly into the source.

% The next lines tell TeXShop to typeset with xelatex, and to open and save the source with Unicode encoding.

%!TEX TS-program = XeLaTeX
%!TEX encoding = UTF-8 Unicode

\usepackage{flupstyleutfbookxelatex}


\usepackage{geometry}                % See geometry.pdf to learn the layout options. There are lots.
%\geometry{a4paper}                   % ... or a4paper or a5paper or ... 
%\geometry{landscape}                % Activate for for rotated page geometry
%\usepackage[parfill]{parskip}    % Activate to begin paragraphs with an empty line rather than an indent
\usepackage{graphicx}
\usepackage{amssymb}

% Will Robertson's fontspec.sty can be used to simplify font choices.
% To experiment, open /Applications/Font Book to examine the fonts provided on Mac OS X,
% and change "Hoefler Text" to any of these choices.

\usepackage[frenchb]{babel}
\usepackage{polyglossia}
\setdefaultlanguage{french}

\usepackage{fontspec,xltxtra,xunicode}
%\defaultfontfeatures{Mapping=tex-text}
%\setromanfont[Mapping=tex-text]{Garaline Regular}
%\setromanfont{Linux Libertine O}
%\setsansfont[Scale=MatchLowercase,Mapping=tex-text]{Gill Sans}
%\setmonofont[Scale=MatchLowercase]{Andale Mono}

%\defaultfontfeatures{Mapping=tex-text, Fractions=On} 
\defaultfontfeatures{Mapping=tex-text} 
\defaultfontfeatures{Ligatures=TeX}
\setromanfont{Linux Libertine O}
\setsansfont{Linux Biolinum O}
%\setromanfont [Ligatures={TeX}, Numbers={OldStyle}, Variant=02]{Linux Libertine O} 
%\setromanfont [Ligatures={Common}, Numbers={OldStyle}, Variant=01]{Linux Libertine O}
%\setromanfont{Adobe Garamond Pro}
%\setsansfont{Linux Biolinum O}

%\usepackage[charter]{mathdesign}




\usepackage{tikz}
%\usepackage{fetatolatex}
\usepackage{lilyglyphs}
\usepackage{musicexamples}
\usepackage{xmpFancyref}
\setXmpListName{Table des exemples musicaux}
\setXmpCaptionLabel{Ex. mus.}



\usepackage{microtype}
\usepackage{multirow}
\usepackage[babel,french=guillemets*]{csquotes}
\MakeOuterQuote{"}
\frenchspacing
\usepackage[hypertexnames=false]{hyperref}
%\newcommand{\ieme}{\textsuperscript{ieme}}
%\newcommand{\iere}{\textsuperscript{iere}}
\setlength{\parindent}{0pt} %Supprime l'indentation en début de paragraphe
\usepackage{makeidx}
\FrenchFootnotes
%\usepackage{libertineotf}

% taking care of orphans/widows 
\widowpenalty=10000 
\clubpenalty=10000 
\raggedbottom
\sloppy

%
%
%\usepackage{flupstyleutfxelatex}
%
%\usepackage[frenchb]{babel}
%
%\usepackage{xunicode,fontspec,xltxtra}
%
%
%
%\usepackage{polyglossia}                                
%\setdefaultlanguage{french}          
%
%
\usepackage[parfill]{parskip}
%
%\setmainfont{Linux Libertine O}
%\setromanfont{Linux Libertine O}
%\setsansfont{Linux Biolinum O}
%
%\usepackage{url}
%\usepackage{ae}
%%\usepackage{microtype}
%
%%\usepackage{textcomp}
%%\usepackage{pxfonts}
%%\makeatletter
%
%\usepackage{fetatolatex}
%\usepackage{xkeyval}

%\usepackage{scrpage2}
%\pagestyle{scrheadings}

\usepackage[multiple]{footmisc}

%\usepackage[charter]{mathdesign}
%\usepackage{libertine}
%\renewcommand{\bfseries}{\textsb}

%\usepackage[colorlinks=false]{hyperref}
%\hypersetup{ pdfinfo={
%Title={Concerti grossi},
%Subject={Édition critique - Laure Bellessa}, 
%Author={Corelli - Geminiani}, % ...
%}
%}
\usepackage{makeidx}
\makeindex
\KOMAoptions {listof=totoc}
\KOMAoptions  {index=totoc}
\renewcommand*{\partpagestyle}{empty}


%%%%%SETTINGS BIBLATEX%%%%%

\defbibheading{partitions}{\section*{Partitions}}
\defbibheading{theorie}{\section*{Ouvrages musicaux}}
\defbibheading{divers}{\section*{Ouvrages divers}}
\DeclareFieldFormat[thesis]{citetitle}{\mkbibemph{#1}} % ANCIENNEMENT QUOTE
\DeclareFieldFormat[thesis]{title}{\mkbibemph{#1}} %ANCIENNEMENT: QUOTE
%\renewcommand{\bibnamedash}{------,{ }}
\renewcommand{\bibhang}{0.5cm}

\usepackage{caption}
\addto\captionsfrench{\def\figurename{{\bfseries Fig.}}} % Pour que les titres ne soient plus en smallcaps, soient en gras et soient Fig. au lieu de Figure.
\captionsetup{margin=10pt,font=small,labelfont=bf}

\DeclareGraphicsExtensions{.pdf, .jpg, .tif, .gif}
\AddThinSpaceBeforeFootnotes

\title{TRAVAIL DE RECHERCHE\\ CAPE Lecture à vue-Transposition}
\author{Philippe Massart}

%\pagenumbering{gobble} % supprimer les n°s de pages

%\pagestyle{fancy}
%\fancyhead[LE,RO]{ }
%\fancyhead[CE]{ }
%\fancyhead[CO]{ }
%\fancyfoot[CO]{}
%\pagestyle{headings}
%\pagestyle{plain}
% En-tete
%\lhead{}        \chead{Lecture à vue et langages contemporains}        \rhead{}


%%%%%%Aligner les footnotes sur le début du texte et non de la marge%%%%%
\makeatletter
\long\def\@makefntextFB#1{%
	\ifx\thefootnote\ftnISsymbol
		\@makefntextORI{#1}%
	\else
		\rule\z@\footnotesep
		\setbox\@tempboxa\hbox{\@thefnmark}%
			\ifdim\wd\@tempboxa>\z@
				\kern2em\llap{\@thefnmark.\kern0.5em}%
			\fi
		\hangindent2em\hangafter\@ne#1
	\fi}
\makeatother
%%%%%%%%%%%%%%%%%%%%%%%%%%%%%%%%%%%%%%%%%

\begin{document}
\pagestyle{empty}

%%\titlepage
%\date {26 avril 2008}
%%\maketitle
%\begin{center}
%\begin{LARGE} \textbf{CONCERTI GROSSI\\ \vspace{1cm}
%%TITRE (suite)
%} 
%\end{LARGE}  \vspace{1cm}
% \textit{con due violini, viola e violoncello di concertino obligati}
% \\ \textit{e due altri violini e basso di concerto grosso} 
%  \vspace{1cm}
%\\ \textbf{da}
%\vspace{1cm}
%\\ \begin{large} \textbf {Francesco Geminiani} \end{large}
%\vspace{1.5cm}
%\\ Édition critique d'après la copie de la Bibliothèque musicale François Lang de Royaumont \vspace{9cm}
%\\ Laure \textsc{Bellessa}\vspace{1cm}
%\\ Master universitaire Musique : Recherche et Pratiques d’Ensemble
%%\\Master machin (suite)
%\end{center}
%
%\clearpage
%
%\onehalfspacing
%\tableofcontents
%\setcounter{secnumdepth}{-1}
%\thispagestyle{empty}

\newpage

It was intended that this piece be played several times, and differently each time, returning from the mark D.C. and playing the Coda; as is customary, the last time; but each time all strings play the measures before the bar B.C.

\begin{center}
\textbf{If played four time}
\end{center}

\textbf{First time} Allegretto and \lilyDynamics{pp}. Second violin and cello only, until two measures before D.C. which all strings play each time. No piano.

\textbf{Seconde time} Allegro moderato --- \lilyDynamics{mp}. First violin and viola only, until two measures before D.C. No piano.

\textbf{Third time} Allegro molto --- strings \lilyDynamics {f} but piano {p}. All strings and piano, which plays only outer notes of each chord, that is, the upper and lower notes only, in each hand.

\textbf{Fourth time} Presto (or as fast as possible without disabling any player or instrument) --- double \lilyDynamics{ff}. All play all notes and Coda.



\begin{center}
\textbf{If played three time}
\end{center}

\textbf{First time} Allegretto --- \lilyDynamics{pp}. Second violin and cello, until two measures before D.C. which all play. No piano.

\textbf{Seconde time} Allegro --- \lilyDynamics{mf}. All strings, piano may play; if so --- \lilyDynamics{pp}. Only upper and lower notes in each hand; or piano may not play at all this time.

\textbf{Third time} Presto --- \lilyDynamics{ff}. All play all notes and Coda.

In any case, the playing gets faster and louder each time, keeping up with the bonfire.

It has been observed by friends that three times around is quite enough while others stood for four --- but as this piece was written for a Haloowe'en party and not for a nice concert, the decision must be made by the players, regardless of the feelings of the audience.

\textbf{P.S.} A bass drum or drum during the last time may play the total rests in measures 3, 4, 5 and 8, and from there on may add his own part --- impromptu, or otherwise.

\flushright{\textbf{\textsc{Charles Edward Ives}}}
%\includepdf[pages=78-80, addtotoc={78, subsection, 2, 1\ier{} mouvement,}]{../lilypond/score.pdf}%%% CONCERTO 5

\includepdf[pages=1-4]{../Ives_Halloween.pdf}

\includepdf{../Ives_Halloween_violinI.pdf}
\includepdf{../Ives_Halloween_violinII.pdf}
\includepdf{../Ives_Halloween_viola.pdf}
\includepdf{../Ives_Halloween_cello.pdf}

\end{document}